\documentclass[12pt]{article}
\usepackage{geometry}
\usepackage{fancyhdr}
\usepackage{graphicx}
\usepackage{titling}
\usepackage{hyperref}

\geometry{a4paper, margin=1in}

\title{User Manual}
\author{
    Brian Becerra\\
    \and
    Jason Salazar\\
    \and
    Krystal Lo\\
    \and
    Mark Canseco\\
    }
\date{April 2025}

\setlength{\droptitle}{6cm}

\pagestyle{fancy}
\fancyhead[L]{User Manual}
\fancyhead[R]{Page \thepage}
\fancyfoot[C]{}

\begin{document}

\begin{titlepage}
\maketitle
\thispagestyle{empty}
\end{titlepage}

\section*{Jira Project Link}

\noindent \textbf{Jira Project URL:} \href{https://calstatela-cs3338-spr25.atlassian.net/}{https://calstatela-cs3338-spr25.atlassian.net/}

\section*{Formal Objective Breakdown}

The primary objective is to implement the foundation for on-field data capture of audit information using a mobile application. This includes:

\begin{itemize}
    \item Developing a user-friendly interface on the Android mobile application for entering key audit data.
    \item Implementing local storage on the mobile device to contain the entered audit data.
    \item Creating a basic display within the mobile application to allow users to view the list of locally saved audit entries.
    \item Setting up the networking components on the mobile application to communicate with the backend server.
    \item Defining a basic API endpoint on the Spring backend to receive audit data from the mobile application.
\end{itemize}

\newpage

\section*{Goals}

The Landfill e-Forms Application aims to modernize the City of Los Angeles, Department of Sanitation's methane emission auditing process by transitioning from a paper-based system to a fully electronic one. This digital transformation offers several key benefits:

\begin{itemize}
    \item \textbf{Improved Efficiency:} Electronic data entry on mobile devices reduces the time and effort required for manual form filling.
    \item \textbf{Reduced Human Error:} Digital forms with built-in validation can minimize data entry errors common with paper-based systems.
    \item \textbf{Faster Report Production:} Electronic data is readily available for analysis and report generation, significantly speeding up the reporting process.
    \item \textbf{Real-time Data Access:} The ability to synchronize field data with a central server allows for quicker access to information for all stakeholders.
    \item \textbf{Enhanced Data Management:} Electronic data is easier to store, search, and analyze, leading to better insights and decision-making.
\end{itemize}

\newpage

\section*{How to Download and Access}
\newpage

\end{document}