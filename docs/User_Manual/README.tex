\documentclass[12pt]{article}
\usepackage{geometry}
\usepackage{fancyhdr}
\usepackage{graphicx}
\usepackage{titling}
\usepackage{hyperref}

\geometry{a4paper, margin=1in}

\title{User Manual}
\author{
    Brian Becerra\\
    \and
    Jason Salazar\\
    \and
    Krystal Lo\\
    \and
    Mark Canseco\\
    }
\date{April 2025}

\setlength{\droptitle}{6cm}

\pagestyle{fancy}
\fancyhead[L]{User Manual}
\fancyhead[R]{Page \thepage}
\fancyfoot[C]{}

\begin{document}

\begin{titlepage}
\maketitle
\thispagestyle{empty}
\end{titlepage}

\section*{Jira Project Link}

\noindent \textbf{Jira Project URL:} \href{https://calstatela-cs3338-spr25.atlassian.net/}{https://calstatela-cs3338-spr25.atlassian.net/}

\section*{Formal Objective Breakdown}

The primary objective is to implement the foundation for on-field data capture of audit information using a mobile application. This includes:

\begin{itemize}
    \item Developing a user-friendly interface on the Android mobile application for entering key audit data.
    \item Implementing local storage on the mobile device to contain the entered audit data.
    \item Creating a basic display within the mobile application to allow users to view the list of locally saved audit entries.
    \item Setting up the networking components on the mobile application to communicate with the backend server.
    \item Defining a basic API endpoint on the Spring backend to receive audit data from the mobile application.
\end{itemize}

\newpage

\section*{Goals}

The Landfill e-Forms Application aims to modernize the City of Los Angeles, Department of Sanitation's methane emission auditing process by transitioning from a paper-based system to a fully electronic one. This digital transformation offers several key benefits:

\begin{itemize}
    \item \textbf{Improved Efficiency:} Electronic data entry on mobile devices reduces the time and effort required for manual form filling.
    \item \textbf{Reduced Human Error:} Digital forms with built-in validation can minimize data entry errors common with paper-based systems.
    \item \textbf{Faster Report Production:} Electronic data is readily available for analysis and report generation, significantly speeding up the reporting process.
    \item \textbf{Real-time Data Access:} The ability to synchronize field data with a central server allows for quicker access to information for all stakeholders.
    \item \textbf{Enhanced Data Management:} Electronic data is easier to store, search, and analyze, leading to better insights and decision-making.
\end{itemize}

\newpage

\section*{How to Download and Access}
\subsection{Web Application}
To launch the front-end server, you will need to have Node and npm installed. You will also need to have Angular-CLI installed. A TypeScript editor is needed, such as Visual Studio Code.\newline
Assuming Visual Studio Code is used, open the "landfill-web-app/client" folder.  Run the command ng serve to start the front-end server. The home page can then be accessed at http://localhost:4200/.\newline \newline
To launch the back-end server, you will need to have Java 8 (JDK or OpenJDK recommended),  Eclipse or another Java IDE, and the Gradle Buildship plugin for Eclipse, or the equivalent for other IDEs, are required.\newline
Start Eclipse and import "landfill-web-app/server" as a Gradle Project. If Gradle has not automatically downloaded the project dependencies, right click on the 'server' project folder and click "Gradle $\rightarrow$ Refresh Gradle Project". Add the Microsoft SQL Server JDBC driver "sqljdbc42.jar" from the "server/lib" directory to the project's build path.\newline
To ensure that the proper 'dev' profile is loaded when running an instance of the back-end server on your machine, add the parameter "--spring.profiles.active=dev" to the run configuration in Eclipse. Note that the back-end server is set to listen on port 8080 for when running in development mode, so make sure that no other processes are listening on the same port number.
\subsection{Mobile Application}
To launch the mobile application, you will need to have Android Studio installed and Java 8 (JDK recommended). The use of another IDE for development will require the Android SDK to be installed seperately. Make sure that the Android SDK is installed and configured properly.\newline
Open Android Studio and import the "landfill-web-app/mobile" folder as a new project. The Gradle build system will automatically download the project dependencies. \newline
A phone running Android 5.1(Lollipop) or higher is required to run the application. The emulator can be used, but it is recommended to use a physical device for testing. Make sure that the device is connected to the same network as the back-end server.\newline
If installing the application through Android Studio, connect the device to the computer and ensure the device is recognized. Go to "Build $>$ Build APK", which will install the application on the device.\newline
Please note that after any updates to the databse, the application data must be cleared to avoid the application crashing. This can be done by going to the Application Manager on the device, selecting the Landfill e-Forms Application, and selecting "Clear All" application data.\newline
If installing the application through file download, make sure there is no device connected to the computer. Go to "Build $>$ Build APK", which will create an APK file in the "landfill-web-app/mobile/app/build/outputs/apk" folder. Transfer the APK file to the device and open it to install it. 

\end{document}